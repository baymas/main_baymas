\documentclass{ctexart}

\usepackage{graphicx}
\usepackage{amsmath}
\usepackage[textwidth=14.5cm]{geometry}
\usepackage{blindtext}
\parindent=0pt

\bibliographystyle{plain}
\title{Homework Three}


\author{颜铂林 \\ 数学与应用数学 3210101536}

\begin{document}
\begin{sloppypar}

\maketitle


导言:这是一篇关于我的Linux工作环境的说明文。

\section*{Linux发行版本与版号}

打开terminal,输入以下命令\verb!sudo lsb_release -a!,返回:

\begin{verbatim}

No LSB modules are available.
Distributor ID:	Ubuntu
Description:	Ubuntu 18.04.6 LTS
Release:	18.04
Codename:	bionic

\end{verbatim}

得到发行版本是Ubuntu 18.04.6 LTS,版本号是18.04。

\section*{环境配置}

\textbf{Step1:}打开Software\&Updates,在Download from一栏选择了Other,然后挑选\textbf{mirrors.cn99.com}作为下载源。打开shell,输入\verb|sudo apt-get install synaptics|安装\textbf{synaptics} 。

\textbf{Step2:}打开synaptics,安装\textbf{fcitx}和\textbf{fcitx-googlepinyin},在language support中将输入法更换为fcitx,并且添加googlepinyin。

\textbf{Step3:}打开synaptics,安装\textbf{g++, gcc, make, cmake, automake, emacs, vim, gedit, texlivefull, doxygen, doxygen.doc, doxymacs, libboost-all-dev, trilinos-all-dev ,trilions-dbg, trilions.doc, dx, git, ssh, vnc4server, x11vnc}

\textbf{Step4:}打开shell,输入\verb|emacs .bashrc|进入emacs的配置文件,在其中添加\verb|alias emacs='LC.CTYPE=zh_CN.utf8 emacs'|,使得在emacs中可以输入中文。

\section*{未来规划}

\textbf{PART1}

预计将会在之后的数据结构,数值分析等课程里面继续使用Linux环境。并且在今后编写数学文章时会在Linux中借助emacs编辑latex文档。

\textbf{PART2}

我认为当前环境不太符合未来需求。由于这是一台刚配置好不超过三天的虚拟机,其中许多功能还未完善。

1.缺少工作系统中的代码管理,以及文献管理,需要熟练使用git以长期稳定保存代码,创建云盘以实时备份文件。

2.emacs编辑器缺少适合自己习惯的专属配置,这一点将在今后使用过程中逐步完善解决。目前借鉴老师和助教的配置方案来暂时满足工作条件。

3.缺少一些脚本文件来辅助提高工作效率,将在之后按照自己的需求逐渐增加。

\section*{工作结果管理}

在阅读了\textbf{Github: An Introduction}\cite{2016Github}以及\textbf{Social coding in GitHub: transparency and collaboration in an open software repository}\cite{2012Social}之后,我了解到了使用github管理代码的优越之处,并且注册了github帐号,配置好了git环境用来管理我的代码。

而在老师的推荐下,决定使用坚果云来实时保存工作中的重要文件和工作结果。


\bibliography{books}




\end{sloppypar}
\end{document}
